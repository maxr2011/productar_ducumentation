\documentclass{scrartcl}

\usepackage[utf8]{inputenc}
\usepackage{color}
\usepackage{xcolor}

\title{Dokumentation ProductAR}
\author{Maximilian Rehberger}
\date{June 2019}

\setcounter{secnumdepth}{4}
\setcounter{tocdepth}{4}

\definecolor{darkcerulean}{rgb}{0.03, 0.27, 0.49}
\definecolor{frenchblue}{rgb}{0.0, 0.45, 0.73}
\definecolor{babyblueeyes}{rgb}{0.63, 0.79, 0.95}

\addtokomafont{section}{\color{darkcerulean}}
\addtokomafont{subsection}{\color{frenchblue}}
\addtokomafont{subsubsection}{\color{babyblueeyes}}

\begin{document}


\maketitle

\newpage


\renewcommand*\contentsname{}
\section{Inhaltsverzeichnis}
\tableofcontents{}


\newpage

\section{Einleitung}

\subsection{Zweck}


\newpage

\section{Allgemeine Übersicht}

\subsection{Beschreibung Ausgangssituation}

\subsection{Produkteinsatz}

\subsection{Produktumfeld}

\subsection{Produktfunktionalität}

\subsection{Personas}

\subsubsection{Nutzer}

\subsubsection{Verkäufer}

\subsubsection{Admin}


\newpage

\section{Architekturkonzept und Entwurf}

\subsection{Ursprüngliches Architekturkonzept}

\subsection{Anfängliche Skizze Datenbankentwurf}

\subsubsection{MySQL Datenkbank (Remote)}

\subsection{Anfängliche Skizze Java Klassen}

\subsection{Endgültige Skizze Datenbankentwurf}

\subsubsection{SQLite Datenbank (Lokal)}
todo: erzeugen

\subsubsection{MySQL Datenbank (Remote)}
todo: erzeugen

\subsection{Endgültige Skizze Java Klassen}
todo: erzeugen

\subsection{Übersicht Backend Server}

\subsection{Übersicht REST API}

\subsection{Technische Entscheidungen}

\subsubsection{Warum Android?}

\subsubsection{Welche Androidversion?}

\subsubsection{Welche Entwicklungsumgebung?}

\subsubsection{Warum eine MySQL Datenbank?}

\subsubsection{Warum eine REST API?}

\subsubsection{Vergleich mit Alternativlösungen}

\paragraph{Firebase von Google}

\paragraph{Alternative Datenbankmodelle}


\newpage

\section{Technische Dokumentation}

\subsection{Android Manifest}

\subsection{Java Interfaces}

\subsection{Java Klassen}

\subsubsection{Objekt Klassen}

\paragraph{Object Class (Abstract)}

\paragraph{Product}

\paragraph{User}

\paragraph{Model}

\paragraph{Photo}

\paragraph{Price}

\paragraph{Shop}

\paragraph{Category (Enum)}

\paragraph{Currency (Enum)}

\paragraph{Interval (Enum)}

\subsubsection{Aktivity Klassen}

\paragraph{MainActivity}

\paragraph{SplashScreen}

\paragraph{ProductArActivity}

\paragraph{ProductScanActivity}

\paragraph{CaptureActivityPortrait}

\paragraph{LastScannedProductsActivity}

\paragraph{CreateProductActivity}

\paragraph{ProductDetailActivity}

\paragraph{ProductPhotoGalleryActivity}

\paragraph{ProductPhotoDetailActivity}

\paragraph{CreatePriceActivity}

\paragraph{PriceHistoryActivity}

\paragraph{RegisterActivity}

\paragraph{LoginActivity}

\paragraph{ProfileActivity}

\paragraph{SettingsActivity}

\paragraph{InfoActivity}

\subsubsection{Adapter Klassen}

\paragraph{ProductListAdapter}

\paragraph{PhotoAdapter}

\subsubsection{Hilfs Klassen}

\paragraph{GeneralHelper}

\paragraph{BarcodeHelper}

\paragraph{QRCodeHelper}

\paragraph{LoginHelper}

\paragraph{SettingsHelper}

\paragraph{ImageHelper}

\paragraph{PhotoHelper}

\paragraph{UploadHelper}

\paragraph{PriceHelper}

\subsubsection{Fragment Klassen}

\paragraph{ScanFragment}

\paragraph{CustomArFragment}

\subsection{Retrofit Schnittstelle}

\subsection{Network Monitor}

\subsection{Background Service}

\subsection{Notifications}

\subsection{Ressourcen}

\subsubsection{Layout}

\subsubsection{Drawable Icons}

\subsubsection{App Icon}

\subsubsection{Animation}

\subsubsection{Menu}

\subsubsection{Assets}

\subsubsection{Values}

\subsection{Rest Api}


\newpage

\section{Veröffentlichung im Google Play Store}

\subsection{Store Eintrag}

\subsection{Alpha Test}

\subsection{Beta Test}


\newpage

\section{Zukünftige Entwicklungen}


\newpage

\section{Fazit}


\newpage

\section{Verwendete Frameworks und Software}


\newpage

\section{Verlinkung Repositories}


\newpage

\section{Verlinkung Tutorials}


\newpage

\section{Quellenangabe}


\end{document}
