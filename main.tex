\documentclass{scrartcl}

\usepackage[utf8]{inputenc}
\usepackage{color}
\usepackage{xcolor}

\title{Dokumentation ProductAR}
\author{Maximilian Rehberger}
\date{June 2019}

\setcounter{secnumdepth}{4}
\setcounter{tocdepth}{4}

\definecolor{darkcerulean}{rgb}{0.03, 0.27, 0.49}
\definecolor{frenchblue}{rgb}{0.0, 0.45, 0.73}
\definecolor{babyblueeyes}{rgb}{0.63, 0.79, 0.95}

\addtokomafont{section}{\color{darkcerulean}}
\addtokomafont{subsection}{\color{frenchblue}}
\addtokomafont{subsubsection}{\color{babyblueeyes}}

\begin{document}


\maketitle

\newpage


\renewcommand*\contentsname{}
\section{Inhaltsverzeichnis}
\tableofcontents{}


\newpage

\section{Einleitung}

\subsection{Zweck}

\newpage

\section{Allgemeine Übersicht}

\subsection{Beschreibung Ausgangssituation}

\subsection{Produkteinsatz}

\subsection{Produktumfeld}

\subsection{Produktfunktionalität}

\subsection{Personas}

\subsubsection{Nutzer}

\subsubsection{Verkäufer}

\subsubsection{Admin}

\newpage

\section{Architekturkonzept und Entwurf}

\subsection{Ursprüngliches Architekturkonzept}

\subsection{Anfängliche Skizze Datenbankentwurf}

\subsection{Anfängliche Skizze Java Klassen}

\subsection{Endgültige Skizze Datenbankentwurf}
todo: erzeugen

\subsection{Endgültige Skizze Java Klassen}
todo: erzeugen

\subsection{Übersicht Backend Server}

\subsection{Übersicht REST API}

\subsection{Technische Entscheidungen}

\subsubsection{Warum Android?}

\subsubsection{Welche Androidversion?}

\subsubsection{Warum eine MySQL Datenbank?}

\subsubsection{Warume eine REST API?}

\subsubsection{Vergleich mit weiteren Datenabankmodellen}

\newpage

\section{Technische Dokumentation}

\subsection{Java Klassen}

\subsubsection{Objekt Klassen}

\paragraph{Product}

\subsubsection{Aktivity Klassen}

\subsection{Ressourcen}

\subsubsection{Layout}

\subsubsection{Drawable Icons}

\subsubsection{App Icon}

\subsubsection{Animation}

\subsubsection{Menu}


\subsection{Rest Api}

\newpage

\section{Veröffentlichung im Google Play Store}

\subsection{Store Eintrag}

\subsection{Alpha Test}

\subsection{Beta Test}

\newpage

\section{Zukünftige Entwicklungen}

\newpage

\section{Fazit}

\newpage

\section{Verwendete Frameworks und Software}

\newpage

\section{Verlinkung Tutorials}

\newpage

\section{Quellenangabe}


\end{document}
